\newcommand{\betterFormula}{

\begin{equation}
    f_N(t) = \begin{bmatrix}
        x_N(t) \\ y_N(t)
    \end{bmatrix} =
    \begin{bmatrix}
        t \\ f(t)
    \end{bmatrix} +
    N \dfrac{\langle -f'(t), 1 \rangle}{|\langle -f'(t), 1 \rangle|}
\end{equation}

}

\section{A Better Formula}

Next up, let’s establish rigorously a \textit{much} nicer parametric formula for the $N$-Units Away Curves. I figured this out after taking Multivariable Calc in college, and it’s a \textit{huge} improvement over my original high school formula.

Pick some $x_o \in \mathbb{R}$. There is exactly one point on the curve $y=f(x)$ corresponding to that location, namely $(x_o , f(x_o))$. A nice simple vector pointing in the direction of the tangent line to $y=f(x)$ at $x = x_o$ is $\langle 1, f'(x_o) \rangle$. Thus a nice simple vector
pointing in the direction of the normal line to $y = f(x)$ at $x = x_o$ is $\langle -f'(x_o), 1 \rangle$.

We want to establish a vector which points ``up'' along that normal line, but note that now unlike before we do not need to switch signs based on the sign of $f'(x_o)$! Also note that there is no longer any divide by zero issue encountered when $f'(x_o) = 0$. Nice! We want to have a unit normal vector: $\dfrac{\langle -f'(x_o), 1 \rangle}{|\langle -f'(x_o), 1 \rangle|}$ and then multiply this by N to get a vector that literally points from $(x_o, f(x_o))$ to a spot $N$ units away and in the ``up'' direction of the normal line to $y=f(x)$ at $x = x_o$. This gives us a \textit{vastly} new and improved version of the formula for $N$-Units Away Curves: \betterFormula