% define some variables
\newcommand\f{ f'(x_o) }
\newcommand\mfrac{ \dfrac{N}{|\f|} \f }
\newcommand\ang{ tan^{-1} \left( -\dfrac{1}{\f} \right) }
\newcommand\mcos{ cos \left( \ang \right) }
\newcommand\msin{ sin \left( \ang \right) }

% defining a new command for possible future use
\newcommand{\firstFormula}{

  \begin{equation} \tag{1} %  hard-set the number of this formula to be 1
    f_N(t) =
    \begin{bmatrix}
      x_N(t) \\
      y_N(t)
    \end{bmatrix} =
    \begin{cases}
      % case 1: f'(t) is non-0
      % expression
      \begin{bmatrix} % \mfrac, \ang, \mcos, \msin are defined above
        t - \mfrac \mcos \\
        f(t) + \mfrac \msin
      \end{bmatrix}, &
      % condition
      t \ni f'(t) \neq 0 \\
      % case 2: f'(t) is 0
      % expression
      \begin{bmatrix}
        t \\
        f(t) + N
      \end{bmatrix}, &
      % condition
      t \ni f'(t) = 0
    \end{cases}
  \end{equation}

}

\section{A First Formula}

Let $f: \mathbb{R} \to \mathbb{R}$ be any arbitrary twice differentiable function whose second derivative $f''$
is continuous. These are the kinds of functions this paper we’ll use to establish the science of the $N$-Units Away Curves. While you can talk about the building of an $N$-Units Away Curve from a segment of a function (not on all of $\mathbb{R}$), it’s very similar and just not as interesting. Likewise you can also talk about making $N$-Units Away Curves for functions that are non-differentiable or discontinuous, but in that case things often get very nasty very quickly.

Take your function $y = f(x)$ and we're now going to re-express that same function as a parametric function in terms of a variable $t$: \begin{equation*}
  \begin{bmatrix}
    x(t) \\
    y(t)
  \end{bmatrix} =
  \begin{bmatrix}
    t \\
    f(t)
  \end{bmatrix}
\end{equation*}, where $t$ assumes all values $t \in (- \infty, \infty)$. Pick any value $t_o \in \mathbb{R}$. $t_o$ corresponds to one unique point on the original curve $y = f(x)$, specifically the point $(t_o , f(t_o ))$. Fix an $N \in \mathbb{R}$. To every one of these points we are going to construct an associated point that lies exactly $N$ units away in the direction of the normal line to the curve at that point. Let $x_o = t_o$ . It’s a basic Calc 1 fact that at $x = x_o$ , the slope of the normal line is $-\dfrac{1}{f'(x_o)}$, assuming $f'(x_o)$ is non-zero. We can use arctangent to convert from slope to degrees. We get that the angle from the positive $x$-axis to the normal line is $\ang$.

We can then use $cos(\theta)$ to convert this angle to an $x$-distance and $sin(\theta)$ to convert this angle to a $y$-distance. If we multiply these both by $N$ we can obtain a normal vector $\langle N \mcos, \allowbreak N \msin \rangle $ which is exactly $N$ units long and points in the direction of the normal line to the curve $y = f(x)$ at $x = x_o$. The only last issue that causes trouble is the issue of which way we are trying to travel on that normal line? Let’s define it such that when N is positive, we obtain points $N$ units away and above the curve $y = f(x)$, and when $N$ is negative, we obtain points $N$ units away and below the curve. When $f'(x_o)$ is negative, the normal vector stated above points up and to the right, so we’re fine. It points to a spot above the curve. When $f'(x_o)$ is positive however, the normal vector points down and to the right, so we must multiply the whole vector by -1. This resolves the issue. We arrive at the desired normal vector that points $N$ units away into the ``up'' direction along the normal line to $y = f(x)$ at $x = x_o$. That vector is $\langle - \mfrac \mcos, - \mfrac \msin \rangle$.

We can utilize this to derive a parametric formula for the $N$-Units Away Curve to $y = f(x)$
in all cases where $f'(x_o) \neq 0$. But what about when is equal to 0? It’s nice and simple then.
We simply move $N$ units along a perfectly vertical normal line.

Thus the formula for the $N$-Units Away Curve to $y = f(x)$ is:

\firstFormula
